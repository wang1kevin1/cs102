\documentclass[11pt]{article}
\usepackage{fullpage,amsthm,amsfonts,amssymb,epsfig,amsmath,times,algorithm,algorithmic}
\usepackage[table,xcdraw]{xcolor}

\newtheoremstyle{indented-remark}
{}
{}
{\addtolength{\leftskip}{2.5em}}
{}
{\bfseries}
{:}
{.5em}
{}

\newtheoremstyle{indented-proof}
{}
{}
{\addtolength{\leftskip}{2.5em}}
{}
{\slshape}
{.}
{.5em}
{}

\theoremstyle{definition}
\newtheorem{theorem}{Theorem}
\newtheorem{lemma}{Lemma}
\newtheorem{corollary}{Corollary}
\newtheorem{observation}{Observation}
\newtheorem{definition}{Definition}

\theoremstyle{plain}
\newtheorem{claim}{Claim}

\theoremstyle{indented-remark}
\newtheorem{case}{Case}

\theoremstyle{indented-proof}
\newtheorem*{proofofcase}{Proof of Case}

\begin{document}

\begin{center}
{\bf\Large CMPS 102 --- Fall 2018 ---  Homework 2}
\end{center}

\begin{center}
\textit{"I have read and agree to the collaboration policy." - \textbf{Kevin Wang}}
\end{center}

\section*{Solution to Problem 2: Finding the Peak}

Let $A[1:n]$ be an array of $\mathbb{Z}^{+}$, where $n \geq 3$. Array $A$ has special properties: $A[1] \leq A[2]$ and $A[n-1] \geq A[n]$. We define $A[x]$ as a \textit{peak} if $A[x] \geq A[x-1]$ and $A[x] \geq A[x+1]$.

\begin{algorithm}
\caption{Uses divide-and-conquer to locate a peak in an array}
\begin{algorithmic} 
\STATE \textbf{FIND-PEAK} ($A$ , $n$):
\STATE Let $x = \frac{n}{2}$
\IF{$A[x-1] \leq A[x] \leq A[x+1]$}
\STATE $A[x]$ is a peak. Return index $x$.
\ELSIF{$A[x-1] \geq A[x]$}
\STATE FIND-PEAK ($A[0:x-1]$ , $x$)
\ELSE 
\STATE $\{ \text{When } A[x+1] \geq A[x] \}$
\STATE FIND-PEAK ($A[x:n]$ , $x$)
\ENDIF
\end{algorithmic}
\end{algorithm}

\begin{claim}
The time complexity of the algorithm is $O(\log n)$.
\end{claim}

\begin{proof}
It takes time $O(2)$ to check the neighbors of an array element. The time complexity of each reduced half array is $T\big( \frac{n}{2} \big)$.
Therefore, the recurrence relation for the algorithm is defined as:
\begin{center}
$T(n) = T\big( \frac{n}{2} \big) + O(2)$
\end{center}
Thus, by Case 2 of the Master Theorem, the algorithm's time complexity is $O(\log n)$.
\end{proof}

\end{document}

