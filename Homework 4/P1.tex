\documentclass[11pt]{article}
\usepackage{fullpage,amsthm,amsfonts,amssymb,epsfig,amsmath,times,algorithm,algorithmic}
\usepackage[table,xcdraw]{xcolor}

\setlength{\parindent}{2.5em}

\newtheoremstyle{indented-remark}
{}
{}
{\addtolength{\leftskip}{2.5em}}
{}
{\bfseries}
{:}
{.5em}
{}

\newtheoremstyle{indented-proof}
{}
{}
{\addtolength{\leftskip}{2.5em}}
{}
{\slshape}
{.}
{.5em}
{}

\theoremstyle{definition}
\newtheorem{theorem}{Theorem}
\newtheorem{lemma}{Lemma}
\newtheorem{corollary}{Corollary}
\newtheorem{observation}{Observation}
\newtheorem{definition}{Definition}

\theoremstyle{plain}
\newtheorem{claim}{Claim}

\theoremstyle{indented-remark}
\newtheorem{case}{Case}

\theoremstyle{indented-proof}
\newtheorem*{proofofcase}{Proof of Case}

\begin{document}

\begin{center}
{\bf\Large CMPS 102 --- Fall 2018 ---  Homework 4}
\end{center}

\begin{center}
\textit{"I have read and agree to the collaboration policy." - \textbf{Kevin Wang}}
\end{center}
\begin{center}
{\footnotesize \textbf{Collaborators:} \textit{None}} 
\end{center}

\section*{Solution to Problem 1: Coffee Shops}

Given $n$ minutes to study, Charlie has two sequences $V=\{ v_{1} , v_{2} , \cdots , v_{i} , \cdots , v_{n} \}$ and $R=\{ r_{1} , r_{2} , \cdots , r_{i} , \cdots , r_{n} \}$ which represents the work he can do at the $i$-th minute at Valve and Ruru, respectively. It costs Charlie 10 minutes to switch coffee shops. \\

\noindent Let $W(n)$ be the maximum total work Charlie can do.

\subsection*{Sub-Problems}

\textbf{$\boldsymbol{W_{V}[i]}$ is the max total work done by time $\boldsymbol{i}$ when currently at Valve.} At time $i$ while at Valve, Charlie could have: (1) been studying there already or (2) just arrived from Ruru. \\

\indent \textbf{Base Case:} Charlie has done no work before 9AM. $\longrightarrow W_{V}[i \leq 0]=0$ 


\indent \textbf{Case 1:} Charlie has been studying there already.
\begin{align}
W_{V}[i] &= \text{max work done a minute ago} + \text{work done at the i-th minute at Valve} \notag \\
&= W_{V}[i-1]+v_{i} \notag
\end{align}

\indent \textbf{Case 2:} Charlie just arrived from Ruru.
\begin{align}
W_{V}[i] &= \text{max work done when at Ruru} + \text{work done at the i-th minute at Valve} \notag \\
&= W_{R}[i-10]+v_{i} \notag
\end{align}


\indent Therefore, the max total work done by time $i$ when currently at Valve is the max of cases 1 and 2.
\begin{center}
$\longrightarrow W_{V}[i]=$ max( $W_{V}[i-1]+v_{i}$ , $W_{R}[i-10]+v_{i}$ )
\end{center}

\noindent Similarly, \textbf{$\boldsymbol{W_{R}[i]}$ is the max total work done by time $\boldsymbol{i}$ when currently at Ruru.} \\

\indent \textbf{Base Case:} Charlie has done no work before 9AM. $\longrightarrow W_{R}[i \leq 0] =0$ \\

\indent \textbf{Case 1:} Charlie has been studying there already. $\longrightarrow W_{R}[i]=W_{R}[i-1]+r_{i}$ \\

\indent \textbf{Case 2:} Charlie just arrived from Valve. $\longrightarrow W_{R}[i]=W_{V}[i-10]+r_{i}$

\begin{center}
$\longrightarrow W_{R}[i]=$ max( $W_{R}[i-1]+r_{i}$ , $W_{V}[i-10]+r_{i}$ )
\end{center}

\pagebreak

\noindent Therefore, the maximum work Charlie can complete in $n$ minutes past 9AM is:
\[
\text{W}(n) =
\begin{cases}
0 & \text{if $n \leq 0$}\\
\text{max($W_{V}[n]$,$W_{R}[n]$)}  & \text{otherwise}\\
\end{cases}       
\]

\begin{algorithm}
\caption{Finds the max work Charlie can do in $n$ minutes}
\begin{algorithmic} 
\STATE \textbf{MAX-WORK} ($n$):
\STATE Let $W_{V}[i \leq 0]=0$ and $W_{R}[i \leq 0]=0$
\FOR{$i=1$ to $n$}
\STATE $W_{V}[i]=$ max( $W_{V}[i-1]+v_{i}$ , $W_{R}[i-10]+v_{i}$ )
\STATE $W_{R}[i]=$ max( $W_{R}[i-1]+r_{i}$ , $W_{V}[i-10]+r_{i}$ )
\ENDFOR
\STATE W$(n)=$ max($W_{V}[n]$,$W_{R}[n]$)
\STATE
\STATE \textit{*** To find out when Charlie should move coffee shops, just flag when V $\leftrightarrow$ R. ***}
\end{algorithmic}
\end{algorithm}

\noindent \textbf{Time Complexity: $O(n)$} \\
\noindent The for-loops iterates over 2 comparisons, $n$ times, finishing with 1 last comparison. Total time used is $O(2n+1)$.\\

\noindent \textbf{Space Complexity:} \\
\noindent The 2 arrays $V$ and $R$ are each of size $n$. The 2 sequences, {$v_{1}, \cdots , v_{n}$} and {$r_{1}, \cdots , r_{n}$}, also each have size $n$. Total space used is $O(4n)$.

\end{document}