\documentclass[11pt]{article}
\usepackage{fullpage,amsthm,amsfonts,amssymb,epsfig,amsmath,times,algorithm,algorithmic}
\usepackage[table,xcdraw]{xcolor}

\setlength{\parindent}{2.5em}

\newtheoremstyle{indented-remark}
{}
{}
{\addtolength{\leftskip}{2.5em}}
{}
{\bfseries}
{:}
{.5em}
{}

\newtheoremstyle{indented-proof}
{}
{}
{\addtolength{\leftskip}{2.5em}}
{}
{\slshape}
{.}
{.5em}
{}

\theoremstyle{definition}
\newtheorem{theorem}{Theorem}
\newtheorem{lemma}{Lemma}
\newtheorem{corollary}{Corollary}
\newtheorem{observation}{Observation}
\newtheorem{definition}{Definition}

\theoremstyle{plain}
\newtheorem{claim}{Claim}

\theoremstyle{indented-remark}
\newtheorem{case}{Case}

\theoremstyle{indented-proof}
\newtheorem*{proofofcase}{Proof of Case}

\begin{document}

\begin{center}
{\bf\Large CMPS 102 --- Fall 2018 ---  Homework 4}
\end{center}

\begin{center}
\textit{"I have read and agree to the collaboration policy." - \textbf{Kevin Wang}}
\end{center}
\begin{center}
{\footnotesize \textbf{Collaborators:} \textit{None}} 
\end{center}

\section*{Solution to Problem 4: One Edge, One Unit}

Given:
\begin{itemize}
\item a directed graph $G=(V,E)$ with $n$ nodes and $m$ edges
\item a capacity $c_{e} \in \mathbb{Z}^{+}$ on each edge $e \in E$
\item source $s \in V$ and sink $t \in V$
\end{itemize}

\noindent ... assume there is a maximum acyclic $s$--$t$ flow, $f$, in $G$. \\

\noindent Let $H$ be a modified graph of $G$, where a specific edge $e' \in E$ is picked and its capacity $c_{e'}$ reduced by 1 unit. Find the max flow of $H$. \\

\noindent \textbf{Observation:} If we decrease $c_{e'}$ by 1 unit, then the max flow of $H$ is either the same as $f$ or also decreased by 1 unit. \\

\textbf{Case 1:} The flow of edge $e$ was smaller than the capacity of edge $e$: $f_{e} < c_{e}$. \\

\indent \indent The max flow in $G$, $f$, is also the max flow in $H$ because $f_{e'} \leq c_{e'}$ when $c_{e'} = c_{e} - 1$. \\

\textbf{Case 2:} The flow of edge $e$ was equal to the capacity of edge $e$: $f_{e} = c_{e}$. \\

\indent \indent By reducing $c_{e'}$, the flow becomes invalid as $f_{e'} > c_{e'}$ when $c_{e'} = c_{e} - 1$. \\

\noindent To fix the invalidity, we reduce $f_{e'}$ by 1 unit as well. Then find a simple $u$--$s$ path and a simple $t$--$v$ path and route 1 unit of flow along the path $t$--$v$--$u$--$s$ to balance the in-flow and out-flow of $e'$. The new flow  is 1 unit less than $f$ and is in $H$ as well.

\noindent \textbf{Observation:} An augumenting path from $s$--$t$ has a flow of at most 1 unit. \\

\noindent Finally, we check if there are any augmenting paths from $s$--$t$ using \textbf{Ford-Fulkerson}.\\

\textbf{Case 1:} There is an augmenting path, $p$, with max flow $f_{p} = 1$ \\

\indent \indent We can increase the flow of the edges by 1 unit, resulting in the same max-flow as $G$. \\

\textbf{Case 2:} No augmenting path exists. \\

\indent \indent The max flow of $H$ is less than $f$ by 1 unit. \\

\pagebreak

\begin{algorithm}
\caption{Finds the max flow $f'$ of $H$}
\begin{algorithmic} 
\STATE \textbf{MAX-FLOW}:
\IF{$f_{e} < c_{e}$}
\STATE Return $f$
\ELSE
\STATE \textit{[If: $f_{e} = c_{e}$]}
\STATE Initialize the residual network $H_{f'}$
\STATE Search for augumenting path $p$ (Ford-Fulkerson)
\IF{$p \neq $ null}
\STATE Return $f'+f_{p}$
\ELSE
\STATE Return $f'$
\ENDIF
\ENDIF
\end{algorithmic}
\end{algorithm}

\noindent \textbf{Time Complexity: $O(m+n)$} \\
\noindent Initializing the residual graph takes time $O(m+n)$. Ford Fulkerson runs in time $O(C \cdot (m+n))$ where $C$ is the capacity of the cut from the source node. Total run time is $O((C+1) \cdot (m+n))$. \\

\noindent \textbf{Space Complexity: $O(m+n)$} \\
\noindent Typical space complexity for a graph is $O(m+n)$. 

\end{document}