\documentclass[11pt]{article}
\usepackage{fullpage,amsthm,amsfonts,amssymb,epsfig,amsmath,times,algorithm,algorithmic}
\usepackage[table,xcdraw]{xcolor}

\setlength{\parindent}{2.5em}

\newtheoremstyle{indented-remark}
{}
{}
{\addtolength{\leftskip}{2.5em}}
{}
{\bfseries}
{:}
{.5em}
{}

\newtheoremstyle{indented-proof}
{}
{}
{\addtolength{\leftskip}{2.5em}}
{}
{\slshape}
{.}
{.5em}
{}

\theoremstyle{definition}
\newtheorem{theorem}{Theorem}
\newtheorem{lemma}{Lemma}
\newtheorem{corollary}{Corollary}
\newtheorem{observation}{Observation}
\newtheorem{definition}{Definition}

\theoremstyle{plain}
\newtheorem{claim}{Claim}

\theoremstyle{indented-remark}
\newtheorem{case}{Case}

\theoremstyle{indented-proof}
\newtheorem*{proofofcase}{Proof of Case}

\begin{document}

\begin{center}
{\bf\Large CMPS 102 --- Fall 2018 ---  Homework 4}
\end{center}

\begin{center}
\textit{"I have read and agree to the collaboration policy." - \textbf{Kevin Wang}}
\end{center}
\begin{center}
{\footnotesize \textbf{Collaborators:} \textit{None}} 
\end{center}

\section*{Solution to Problem 2: DNA}

2 DNA sequences $S=\{ s_{1} , s_{2} , \cdots , s_{i} , \cdots , s_{m} \}$ and $T=\{ t_{1} , t_{2} , \cdots , t_{j} , \cdots , t_{n} \}$ have lengths $m$ and $n$ respectively. \\

\noindent Let $LCS$ be the longest common sub-sequence. \\

\noindent \textbf{LCS$\boldsymbol{[i,j]}$ is the length of the largest common sub-sequence, $\boldsymbol{LCS}$, shared by two sequences of sizes $\boldsymbol{i \text{ and } j}$.} \newline

\indent \textbf{Base Case:} An empty sequence does not share a common sub-sequence.
\begin{center}
$LCS[i,0]=0$ and $LCS[0,j]=0$
\end{center}

\indent \textbf{Case 1:} $s_{i} = t_{j}$; the element is a part of the $LCS$.
\begin{center}
$LCS[i,j]=1+LCS[i-1,j-1]$
\end{center}

\indent \textbf{Case 2:} $s_{i} \neq t_{j}$; we ignore one of the elements.
\begin{center}
$LCS[i,j]=$ max( $LCS[i-1,j]$ , $LCS[i,j-1]$ )
\end{center}


\noindent Therefore, the $LCS$ of $S$ and $T$ has length $LCS[m][n]$. We can then walk back through the matrix to find an $LCS$.

\vfill

\begin{center}
Continued...
\end{center}

\pagebreak

\begin{algorithm}
\caption{Finds the $LCS$ of of two sequences $S$ and $T$ with sizes $m$ and $n$, respectively}
\begin{algorithmic} 
\STATE \textbf{LCS} ($m,n$):
\STATE Let $LCS[m][n]$ be the sub-sequence size matrix
\FOR{$i=0$ to $m$}
\FOR{$j=0$ to $n$}
\IF{$i=0$ or $j=0$}
\STATE $LCS[i][j]=0$
\ELSIF{$s_{i} = t_{j}$}
\STATE $LCS[i][j]=LCS[i-1][j-1]+1$
\ELSE
\STATE $LCS[i][j]=$ max( $LCS[i-1][j]$ , $LCS[i][j-1]$ )
\ENDIF
\ENDFOR
\ENDFOR
\STATE $LCS[m][n]$ is the length of the $LCS$
\STATE
\STATE Let $LCS$ be an empty sub-sequence 
\STATE Let $i = m$ and $j = n$
\WHILE{$i>0$ and $j>0$}
\IF{$s_{i} = t_{j}$}
\STATE $s_{i}$ :: $LCS$ 
\STATE $i--$, $j--$
\ELSIF{$LCS[i-1][j] \geq LCS[i][j-1]$}
\STATE \textit{***Using $LCS[i-1][j]>LCS[i][j-1]$ can return another different $LCS$***}
\STATE $i--$
\ELSE 
\STATE $j--$
\ENDIF
\ENDWHILE 
\end{algorithmic}
\end{algorithm}

\noindent \textbf{Time Complexity: $O(mn)$} \\
\noindent The 2 for-loops iterate $m$ and $n$ times, respectively, taking time $O(mn)$. Iterating back through the matrix to find the elements of the $LCS$ takes $O(m+n)$. Total time used is $O(mn + m + n)$.\\

\noindent \textbf{Space Complexity: $O(mn)$} \\
\noindent The $LCS$ matrix takes size $O(mn)$. The LCS sequence, takes (at most) $O($max($m,n$)). Total space used is $O(mn + \text{max($m,n$)})$.

\end{document}