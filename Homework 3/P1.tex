\documentclass[11pt]{article}
\usepackage{fullpage,amsthm,amsfonts,amssymb,epsfig,amsmath,times,algorithm,algorithmic}
\usepackage[table,xcdraw]{xcolor}

\newtheoremstyle{indented-remark}
{}
{}
{\addtolength{\leftskip}{2.5em}}
{}
{\bfseries}
{:}
{.5em}
{}

\newtheoremstyle{indented-proof}
{}
{}
{\addtolength{\leftskip}{2.5em}}
{}
{\slshape}
{.}
{.5em}
{}

\theoremstyle{definition}
\newtheorem{theorem}{Theorem}
\newtheorem{lemma}{Lemma}
\newtheorem{corollary}{Corollary}
\newtheorem{observation}{Observation}
\newtheorem{definition}{Definition}

\theoremstyle{plain}
\newtheorem{claim}{Claim}

\theoremstyle{indented-remark}
\newtheorem{case}{Case}

\theoremstyle{indented-proof}
\newtheorem*{proofofcase}{Proof of Case}

\begin{document}

\begin{center}
{\bf\Large CMPS 102 --- Fall 2018 ---  Homework 3}
\end{center}

\begin{center}
\textit{"I have read and agree to the collaboration policy." - \textbf{Kevin Wang}}
\end{center}

\section*{Solution to Problem 1: Crop Profit}

Let $A[1...n]$ be the array of crop prices in the past $n$ days where $n>2$. $A[i]$ is the crop price on day $i$.

\begin{algorithm}
\caption{Returns the maximum profit possible in the past $n$ days}
\begin{algorithmic} 
\STATE \textbf{MAX-PROFIT} ($A[\ ], n$):
\STATE Initialize index of local minima, $min$
\STATE Initialize index of local maxima, $max$
\STATE Initialize maximum profit, $profit=0$
\STATE Initialize array index, $i=0$
\WHILE{$i < n-1$}
\WHILE{$(i<n-1)$ and $(A[i+1] \leq A[i])$}
\STATE $i++$
\ENDWHILE
\IF{$i == n-1$}
\STATE Break
\ENDIF
\STATE $min=i$, $i++$
\WHILE{$(i<n)$ and $(A[i] \geq A[i-1])$}
\STATE $i++$
\ENDWHILE
\STATE $max=(i-1)$
\STATE $profit += (A[max] - A[min])$
\ENDWHILE
\end{algorithmic}
\end{algorithm}

\begin{claim}
This algorithm is optimal and finds the maximum profit.
\end{claim}

\begin{proof}
Let $s_{i}-b_{i}$ be the profit made from a single sell-buy transaction. The profit of the transactions performed by the algorithm is $(s_{1}-b_{1})+(s_{2}-b_{2})+...$. \newline

\noindent Assume for the sake of contradiction that $b_{1} << s_{2}$ and the transaction $b_{1}$ -- $s_{2}$ is more profitable: \newline
\begin{align}
(s_{2}-b_{1}) - [(s_{1}-b_{1})+(s_{2}-b_{2})] &= -(s_{1}-b_{2})\notag \\
&< 0\notag \\
\implies (s_{2}-b_{1}) &< (s_{1}-b_{1})+(s_{2}-b_{2})\notag
\end{align}
Thus, by contradiction, the algorithm does find the maximum profit.
\end{proof}

\noindent The algorithm goes through the length of $A[1...n]$ and each check incrememnts the index by 1. Thus the algorithm completes the check of all $n$ indices in time: $O(n)$.\newline

\noindent The algorithm requires an array of size $n$ and stores 3 separate values. Thus the space complexity is: $O(n)$.
\end{document}

