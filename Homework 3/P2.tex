\documentclass[11pt]{article}
\usepackage{fullpage,amsthm,amsfonts,amssymb,epsfig,amsmath,times,algorithm,algorithmic}
\usepackage[table,xcdraw]{xcolor}

\newtheoremstyle{indented-remark}
{}
{}
{\addtolength{\leftskip}{2.5em}}
{}
{\bfseries}
{:}
{.5em}
{}

\newtheoremstyle{indented-proof}
{}
{}
{\addtolength{\leftskip}{2.5em}}
{}
{\slshape}
{.}
{.5em}
{}

\theoremstyle{definition}
\newtheorem{theorem}{Theorem}
\newtheorem{lemma}{Lemma}
\newtheorem{corollary}{Corollary}
\newtheorem{observation}{Observation}
\newtheorem{definition}{Definition}

\theoremstyle{plain}
\newtheorem{claim}{Claim}

\theoremstyle{indented-remark}
\newtheorem{case}{Case}

\theoremstyle{indented-proof}
\newtheorem*{proofofcase}{Proof of Case}

\begin{document}

\begin{center}
{\bf\Large CMPS 102 --- Fall 2018 ---  Homework 3}
\end{center}

\begin{center}
\textit{"I have read and agree to the collaboration policy." - \textbf{Kevin Wang}}
\end{center}

\section*{Solution to Problem 2: Space-Efficient MST}

Let connected graph $G=(V,E)$, where $|V|=n$ and $|E|=m$, have a weight function $w:E \rightarrow \mathbb{R}$. \newline
Vertex $s$ is the source.

\begin{algorithm}
\caption{Finds the MWST using eager implementation of Prim's}
\begin{algorithmic} 
\STATE \textbf{EAGER-PRIMS} ($G,s$):
\STATE Initialize array edgeTo[ ] // Stores the shortest edge from a MST vertex
\STATE Initialize array distTo[ ] // Stores weight of edgeTo[ ]
\STATE Initialize array onTree[ ] // Stores 'true' if vertex is on MST
\STATE Initialize indexable priority queue Q
\FOR{$i=1$ to $n$}
\STATE distTo$[i]=\infty$
\ENDFOR
\STATE distTo$[0]=0.0$ 
\STATE $Q = (s$, 0.0) // Initialize indexable queue with key: source weight: 0.0
\WHILE{$|Q| \neq 0$}
\STATE $x=$EXTRACT-MIN($Q$) // Get closest vertex
\STATE onTree$[x]=$ true
\FOR{each $y \in$ adj[$x$]}
\IF{onTree$[y]$}
\STATE Continue
\ELSE
\IF{distTo$\{y\}>$weight$(x$---$y)$}
\STATE edgeTo$[y]$ = $x$---$y$
\STATE distTo$[y]$ = weight$(x$---$y)$
\STATE DECREASE-KEY($y$, distTo$[y]$)
\ENDIF
\ENDIF
\ENDFOR
\ENDWHILE
\end{algorithmic}
\end{algorithm}

\noindent The algorithm uses arrays and a priority queue indexed and keyed by vertex thus each array is at most size $n$. Thus the space complexity is $4 \times O(n)=O(n)$.
\end{document}

