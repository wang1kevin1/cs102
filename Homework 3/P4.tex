\documentclass[11pt]{article}
\usepackage{fullpage,amsthm,amsfonts,amssymb,epsfig,amsmath,times,algorithm,algorithmic}
\usepackage[table,xcdraw]{xcolor}

\newtheoremstyle{indented-remark}
{}
{}
{\addtolength{\leftskip}{2.5em}}
{}
{\bfseries}
{:}
{.5em}
{}

\newtheoremstyle{indented-proof}
{}
{}
{\addtolength{\leftskip}{2.5em}}
{}
{\slshape}
{.}
{.5em}
{}

\theoremstyle{definition}
\newtheorem{theorem}{Theorem}
\newtheorem{lemma}{Lemma}
\newtheorem{corollary}{Corollary}
\newtheorem{observation}{Observation}
\newtheorem{definition}{Definition}

\theoremstyle{plain}
\newtheorem{claim}{Claim}

\theoremstyle{indented-remark}
\newtheorem{case}{Case}

\theoremstyle{indented-proof}
\newtheorem*{proofofcase}{Proof of Case}

\begin{document}

\begin{center}
{\bf\Large CMPS 102 --- Fall 2018 ---  Homework 3}
\end{center}

\begin{center}
\textit{"I have read and agree to the collaboration policy." - \textbf{Kevin Wang}}
\end{center}

\section*{Solution to Problem 4: Cakes}

Given $N$ dollars to spend in a shop that offers $m$ varieties of cakes with distinct prices $S = \{ S_{1} , \cdots , S_{m} \}$, find the possible combinations of cakes given that all money must be spent and that there are unlimited cakes of each variety.

\begin{algorithm}
\caption{Returns the number of cake combinations possible}
\begin{algorithmic} 
\STATE \textbf{CAKE-COMBO} ($S[S_{1} , \cdots , S_{m}], m, N$):
\STATE Initialize $Table[0,\cdots,N][1,\cdots,m]$ // modified indexes
\FOR{$cost=0$ and $variety=1$ to $m$}
\STATE $Table[cost][variety]=1$ 
\ENDFOR
\FOR{$cost=1$ to $N$}
\FOR {$variety=1$ to $m$}
\STATE Let $c_{i}$ be the count of combinations including $S_{variety}$
\STATE Let $c_{e}$ be the count of combinations excluding $S_{variety}$
\STATE $Table[cost][variety]=c_{i}+c_{e}$
\ENDFOR
\ENDFOR
\STATE $Table[N][m]$ contains the total count of combinations possible when spending $N$ dollars on a selection of $m$ varieties
\end{algorithmic}
\end{algorithm}

\noindent The for-loops take time $O(m) + O(N) \cdot O(m)$. Thus the time complexity of this dynamic programming algorithm has time complexity: $O(mN)$. \newline

\noindent The $Table$ takes $O(N) \cdot O(m)$. Thus the time complexity of this dynamic programming algorithm has space complexity: $O(mN)$.

\end{document}