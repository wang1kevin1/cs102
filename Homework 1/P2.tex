\documentclass[11pt]{article}
\usepackage{fullpage,amsthm,amsfonts,amssymb,epsfig,amsmath,times,algorithm,algorithmic}
\usepackage[table,xcdraw]{xcolor}

\newtheoremstyle{indented-remark}
{}
{}
{\addtolength{\leftskip}{2.5em}}
{}
{\bfseries}
{:}
{.5em}
{}

\newtheoremstyle{indented-proof}
{}
{}
{\addtolength{\leftskip}{2.5em}}
{}
{\slshape}
{.}
{.5em}
{}

\theoremstyle{definition}
\newtheorem{theorem}{Theorem}
\newtheorem{lemma}{Lemma}
\newtheorem{corollary}{Corollary}
\newtheorem{observation}{Observation}
\newtheorem{definition}{Definition}

\theoremstyle{plain}
\newtheorem{claim}{Claim}

\theoremstyle{indented-remark}
\newtheorem{case}{Case}

\theoremstyle{indented-proof}
\newtheorem*{proofofcase}{Proof of Case}

\begin{document}

\begin{center}
{\bf\Large CMPS 102 --- Fall 2018 ---  Homework 1}
\end{center}

\begin{center}
\textit{"I have read and agree to the collaboration policy." - \textbf{Kevin Wang}}
\end{center}

\section*{Solution to Problem 2: TA-Course Matching}

Given $m$ courses and $n$ students, we want to assign each student to a TA position in at most one course, in such a way that all TA positions are filled and the matching is stable. \newline

\noindent We assume that $n >$ $\sum_{i=1}^{m} p_{i}$, where $p_{i}$ is the number of positions in course$_{i}$.

\begin{algorithm}
\caption{Perform a stable matching to assign graduate students to TA positions.}
\begin{algorithmic} 
\STATE Initialize each student to be \textit {free}
\STATE Initialize each course[\textit {position}] to be \textit {free}
\WHILE{some student is \textit {free} and has not applied to every course}
\STATE Choose such a student $s$
\STATE $c =$ 1st course on $s$'s list to which $s$ has not applied
\IF{$c$ has a position that is \textit {free}}
\STATE Assign $s$ to a position in $c$
\ELSIF{$c$ prefers $s$ to the least preferred tentative assignment $s'$}
\STATE Assign $s$ to a position in $c$
\STATE Assign $s'$ to be \textit {free}
\ELSE[No positions in $c$ are \textit {free} and $s$ is not preferred more than an existing assignment]
\STATE $c$ rejects $s$ as a TA
\ENDIF
\ENDWHILE
\end{algorithmic}
\end{algorithm}

\begin{observation}
Students apply to courses in decreasing order of preference.
\end{observation}

\begin{observation}
Once a TA position is filled, it never becomes unfilled; it only "trades up" for a more preferred student.
\end{observation}

\begin{observation}
There will always be unassigned students.
\end{observation}

\begin{claim} 
The algorithm terminates after at most $n \times m$ iterations of the while loop.
\end{claim}

\begin{proof} 
Each time the algorithm iterates through the while loop, a student applies to a new course. There are only $n \times m$ possible applications.
\end{proof}

\begin{claim}
All TA positions in all courses are filled.
\end{claim}

\begin{proof}
Suppose, for the sake of contradiction, that a position in course $c$ is not filled upon termination of the algorithm. There is also, by Observation 3, some student $s$, who is not assigned upon termination. By Observation 2, $s$ did not apply to a position in $c$. However, $s$ has applied to every course, since he ends up unassigned -- resulting in a contradiction.
\end{proof} \newpage

\begin{claim}
There are no unstable assignments.
\end{claim}

\begin{proof}
Suppose, for the sake of contradiction, that $c - s'$ is an unstable assignment. 
\begin{case}
The instructor of $c$ favors an unassigned student $s'$ over a current assignment $s$.
\end{case}

\begin{proofofcase}
Assume $s'$ never applied to $c$. By Observation 1, $s'$ prefers their current assignment to $c$. However, $s'$ is unassigned. Thus $c - s'$ is stable.
\end{proofofcase}

\begin{proofofcase}
Assume $s'$ applied to $c$. This implies $s'$ was rejected by the instructor of $c$ immediately or later on. By Observation 2, the instructor of $c$ prefers their current assignment(s) to student $s'$. Thus $c - s'$ is stable.
\end{proofofcase}

\begin{case}
The instructor of $c$ favors student $s'$ over a current assignment $s$ and the student $s'$ favors course $c$ over the current position in course $c'$.
\end{case}

\begin{proofofcase}
Assume $s'$ never applied to $c$. By Observation 1, $s'$ prefers their current assignment $c'$ to $c$. Thus $c - s'$ is stable.
\end{proofofcase}

\begin{proofofcase}
Assume $s'$ applied to $c$. This implies $s'$ was rejected by the instructor of $c$ immediately or later on. By Observation 2, the instructor of $c$ prefers their current assignment(s) to student $s'$. Thus $c - s'$ is stable.
\end{proofofcase}

\noindent In in all possible cases of instability, $c - s'$ is a stable assignment, a contradiction.
\end{proof}

\noindent The total run time of initializing all the students and all the course positions is at most $O(n + (n-1))$, respectively. Running the modified Gale-Shapley takes $O(n*m)$. The time complexity of this algorithm is: $O(n*m)$. \newline

\noindent The total space needed for the array of student assignments and the 2-dimensional array of course position assignments is at most $O(n + (n - 1))$. The spaces needed for both the student preferences of courses and the course instructors rank of students are $O(m*n)$. A queue of unassigned students is at most $O(n)$. The space complexity of this algorithm is: $O(m*n)$.

\end{document}