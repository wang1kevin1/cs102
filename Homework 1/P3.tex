\documentclass[11pt]{article}
\usepackage{fullpage,amsthm,amsfonts,amssymb,epsfig,amsmath,times,algorithm,algorithmic}
\usepackage[table,xcdraw]{xcolor}

\newtheoremstyle{indented-remark}
{}
{}
{\addtolength{\leftskip}{2.5em}}
{}
{\bfseries}
{:}
{.5em}
{}

\newtheoremstyle{indented-proof}
{}
{}
{\addtolength{\leftskip}{2.5em}}
{}
{\slshape}
{.}
{.5em}
{}

\theoremstyle{definition}
\newtheorem{theorem}{Theorem}
\newtheorem{lemma}{Lemma}
\newtheorem{corollary}{Corollary}
\newtheorem{observation}{Observation}
\newtheorem{definition}{Definition}

\theoremstyle{plain}
\newtheorem{claim}{Claim}

\theoremstyle{indented-remark}
\newtheorem{case}{Case}

\theoremstyle{indented-proof}
\newtheorem*{proofofcase}{Proof of Case}

\begin{document}

\begin{center}
{\bf\Large CMPS 102 --- Fall 2018 ---  Homework 1}
\end{center}

\begin{center}
\textit{"I have read and agree to the collaboration policy." - \textbf{Kevin Wang}}
\end{center}

\section*{Solution to Problem 3: Tokens in the Bag}

Given a bag of red, green, and/or blue tokens, we want to prove that performing the following algorithm will terminate:

\begin{algorithm}
\caption{Repeats until the bag of tokens is empty}
\begin{algorithmic} 
\WHILE{there are more than 2 tokens in the bag}
\STATE Take 2 random tokens out of the bag
\IF{one of the tokens is \textit{red}}
\STATE Do nothing
\ELSIF{both tokens are \textit{green}}
\STATE Put 1 \textit{green} token and 2 \textit{blue} tokens back into the bag
\ELSE[At least 1 \textit{blue} token and the other token is not \textit{red}]
\STATE Put 3 \textit{red} tokens back into the bag
\ENDIF
\ENDWHILE { [Exactly 2 or fewer tokens in the bag]}
\STATE Empty the bag
\end{algorithmic}
\end{algorithm}

\noindent We assume that there is always enough tokens to put back into the bag.

\begin{observation}
A selection with a red token decreases the token count of the bag.
\end{observation}

\begin{lemma}
A selection with a blue token results in a decrease of the bag's token count.
\begin{proof}
\begin{case}
If the blue token is selected with a red one, by Observation 1, the token count of the bag is decreased. 
\end{case}
\begin{case}
If the blue token is selected with a green or another blue token, they are replaced by 3 red tokens. By Observation 1, this subset of red tokens will result in a decrease of the bag's token count.
\end{case}
\end{proof}
\end{lemma}

\setcounter{case}{0}
\begin{lemma}
A selection with a green token results in a decrease of the bag's token count.
\begin{proof}
\begin{case}
If the green token is selected with a red one, by Observation 1, the token count of the bag is decreased. 
\end{case}
\begin{case}
If the green token is selected with a blue token, by Lemma 1, the selection results in a decrease of the bag's token count.
\end{case}
\begin{case}
If the green token is selected with another green token, they are replaced by 1 green and 2 blue tokens. By Lemma 1, all possible combinations of 2 tokens selected from this subset result in a decrease of the bag's token count.
\end{case}
\end{proof}
\end{lemma}

\begin{lemma}
All possible color combinations will lead to a decrease of a bag's total token count.
\begin{proof}
By Observation 1, Lemma 1, and Lemma 2, a selection with any color will lead to the eventual removal of the affected subset of tokens.
\end{proof}
\end{lemma}

\setcounter{case}{0}
\begin{claim}
This process will always terminate.
\end{claim}

\begin{proof}
Let $P(n)$ be the statement: "A bag of $n$ tokens will always terminate", where $n$ is the number of tokens in the bag and $n \in \mathbb{N}$. \newline

\noindent It is stated that the bag is emptied -- and thus the program terminated -- when there are exactly 2 or fewer tokens left in the bag prior to a loop. Therefore, the base cases of $P(1)$ and $P(2)$ are true. \newline

\noindent Assume for a bag of $n$ tokens, where $n > n_{0}$ and $n_{0} = 2$, that $P(n)$ is true. We need to prove that $P(n + 1)$ is true. \newline

\noindent Let a bag of $n + 1$ tokens be equivalent to a bag of $n$ tokens plus a single token. Using the induction hypothesis, we assume that the process terminates on the subset of $n$ tokens. By Lemma 3, the additional token -- whether it be red, blue, or green, respectively -- will always result in a net decrease of tokens. We also know that 1 single token left-over will still result in termination. Thus $P(n) \implies P(n + 1)$ is true, proving that the process will always terminate.
\end{proof}

\end{document}